
%% bare_conf.tex
%% V1.3
%% 2007/01/11
%% by Michael Shell
%% See:
%% http://www.michaelshell.org/
%% for current contact information.
%%
%% This is a skeleton file demonstrating the use of IEEEtran.cls
%% (requires IEEEtran.cls version 1.7 or later) with an IEEE conference paper.
%%
%% Support sites:
%% http://www.michaelshell.org/tex/ieeetran/
%% http://www.ctan.org/tex-archive/macros/latex/contrib/IEEEtran/
%% and
%% http://www.ieee.org/

%%*************************************************************************
%% Legal Notice:
%% This code is offered as-is without any warranty either expressed or
%% implied; without even the implied warranty of MERCHANTABILITY or
%% FITNESS FOR A PARTICULAR PURPOSE! 
%% User assumes all risk.
%% In no event shall IEEE or any contributor to this code be liable for
%% any damages or losses, including, but not limited to, incidental,
%% consequential, or any other damages, resulting from the use or misuse
%% of any information contained here.
%%
%% All comments are the opinions of their respective authors and are not
%% necessarily endorsed by the IEEE.
%%
%% This work is distributed under the LaTeX Project Public License (LPPL)
%% ( http://www.latex-project.org/ ) version 1.3, and may be freely used,
%% distributed and modified. A copy of the LPPL, version 1.3, is included
%% in the base LaTeX documentation of all distributions of LaTeX released
%% 2003/12/01 or later.
%% Retain all contribution notices and credits.
%% ** Modified files should be clearly indicated as such, including  **
%% ** renaming them and changing author support contact information. **
%%
%% File list of work: IEEEtran.cls, IEEEtran_HOWTO.pdf, bare_adv.tex,
%%                    bare_conf.tex, bare_jrnl.tex, bare_jrnl_compsoc.tex
%%*************************************************************************

% *** Authors should verify (and, if needed, correct) their LaTeX system  ***
% *** with the testflow diagnostic prior to trusting their LaTeX platform ***
% *** with production work. IEEE's font choices can trigger bugs that do  ***
% *** not appear when using other class files.                            ***
% The testflow support page is at:
% http://www.michaelshell.org/tex/testflow/



% Note that the a4paper option is mainly intended so that authors in
% countries using A4 can easily print to A4 and see how their papers will
% look in print - the typesetting of the document will not typically be
% affected with changes in paper size (but the bottom and side margins will).
% Use the testflow package mentioned above to verify correct handling of
% both paper sizes by the user's LaTeX system.
%
% Also note that the "draftcls" or "draftclsnofoot", not "draft", option
% should be used if it is desired that the figures are to be displayed in
% draft mode.
%
\documentclass[conference]{IEEEtran}
% Add the compsoc option for Computer Society conferences.
%
% If IEEEtran.cls has not been installed into the LaTeX system files,
% manually specify the path to it like:
% \documentclass[conference]{../sty/IEEEtran}





% Some very useful LaTeX packages include:
% (uncomment the ones you want to load)


% *** MISC UTILITY PACKAGES ***
%
%\usepackage{ifpdf}
% Heiko Oberdiek's ifpdf.sty is very useful if you need conditional
% compilation based on whether the output is pdf or dvi.
% usage:
% \ifpdf
%   % pdf code
% \else
%   % dvi code
% \fi
% The latest version of ifpdf.sty can be obtained from:
% http://www.ctan.org/tex-archive/macros/latex/contrib/oberdiek/
% Also, note that IEEEtran.cls V1.7 and later provides a builtin
% \ifCLASSINFOpdf conditional that works the same way.
% When switching from latex to pdflatex and vice-versa, the compiler may
% have to be run twice to clear warning/error messages.






% *** CITATION PACKAGES ***
%
%\usepackage{cite}
% cite.sty was written by Donald Arseneau
% V1.6 and later of IEEEtran pre-defines the format of the cite.sty package
% \cite{} output to follow that of IEEE. Loading the cite package will
% result in citation numbers being automatically sorted and properly
% "compressed/ranged". e.g., [1], [9], [2], [7], [5], [6] without using
% cite.sty will become [1], [2], [5]--[7], [9] using cite.sty. cite.sty's
% \cite will automatically add leading space, if needed. Use cite.sty's
% noadjust option (cite.sty V3.8 and later) if you want to turn this off.
% cite.sty is already installed on most LaTeX systems. Be sure and use
% version 4.0 (2003-05-27) and later if using hyperref.sty. cite.sty does
% not currently provide for hyperlinked citations.
% The latest version can be obtained at:
% http://www.ctan.org/tex-archive/macros/latex/contrib/cite/
% The documentation is contained in the cite.sty file itself.






% *** GRAPHICS RELATED PACKAGES ***
%
\ifCLASSINFOpdf
  % \usepackage[pdftex]{graphicx}
  % declare the path(s) where your graphic files are
  % \graphicspath{{../pdf/}{../jpeg/}}
  % and their extensions so you won't have to specify these with
  % every instance of \includegraphics
  % \DeclareGraphicsExtensions{.pdf,.jpeg,.png}
\else
  % or other class option (dvipsone, dvipdf, if not using dvips). graphicx
  % will default to the driver specified in the system graphics.cfg if no
  % driver is specified.
  % \usepackage[dvips]{graphicx}
  % declare the path(s) where your graphic files are
  % \graphicspath{{../eps/}}
  % and their extensions so you won't have to specify these with
  % every instance of \includegraphics
  % \DeclareGraphicsExtensions{.eps}
\fi
% graphicx was written by David Carlisle and Sebastian Rahtz. It is
% required if you want graphics, photos, etc. graphicx.sty is already
% installed on most LaTeX systems. The latest version and documentation can
% be obtained at: 
% http://www.ctan.org/tex-archive/macros/latex/required/graphics/
% Another good source of documentation is "Using Imported Graphics in
% LaTeX2e" by Keith Reckdahl which can be found as epslatex.ps or
% epslatex.pdf at: http://www.ctan.org/tex-archive/info/
%
% latex, and pdflatex in dvi mode, support graphics in encapsulated
% postscript (.eps) format. pdflatex in pdf mode supports graphics
% in .pdf, .jpeg, .png and .mps (metapost) formats. Users should ensure
% that all non-photo figures use a vector format (.eps, .pdf, .mps) and
% not a bitmapped formats (.jpeg, .png). IEEE frowns on bitmapped formats
% which can result in "jaggedy"/blurry rendering of lines and letters as
% well as large increases in file sizes.
%
% You can find documentation about the pdfTeX application at:
% http://www.tug.org/applications/pdftex





% *** MATH PACKAGES ***
%
%\usepackage[cmex10]{amsmath}
% A popular package from the American Mathematical Society that provides
% many useful and powerful commands for dealing with mathematics. If using
% it, be sure to load this package with the cmex10 option to ensure that
% only type 1 fonts will utilized at all point sizes. Without this option,
% it is possible that some math symbols, particularly those within
% footnotes, will be rendered in bitmap form which will result in a
% document that can not be IEEE Xplore compliant!
%
% Also, note that the amsmath package sets \interdisplaylinepenalty to 10000
% thus preventing page breaks from occurring within multiline equations. Use:
%\interdisplaylinepenalty=2500
% after loading amsmath to restore such page breaks as IEEEtran.cls normally
% does. amsmath.sty is already installed on most LaTeX systems. The latest
% version and documentation can be obtained at:
% http://www.ctan.org/tex-archive/macros/latex/required/amslatex/math/





% *** SPECIALIZED LIST PACKAGES ***
%
%\usepackage{algorithmic}
% algorithmic.sty was written by Peter Williams and Rogerio Brito.
% This package provides an algorithmic environment fo describing algorithms.
% You can use the algorithmic environment in-text or within a figure
% environment to provide for a floating algorithm. Do NOT use the algorithm
% floating environment provided by algorithm.sty (by the same authors) or
% algorithm2e.sty (by Christophe Fiorio) as IEEE does not use dedicated
% algorithm float types and packages that provide these will not provide
% correct IEEE style captions. The latest version and documentation of
% algorithmic.sty can be obtained at:
% http://www.ctan.org/tex-archive/macros/latex/contrib/algorithms/
% There is also a support site at:
% http://algorithms.berlios.de/index.html
% Also of interest may be the (relatively newer and more customizable)
% algorithmicx.sty package by Szasz Janos:
% http://www.ctan.org/tex-archive/macros/latex/contrib/algorithmicx/




% *** ALIGNMENT PACKAGES ***
%
%\usepackage{array}
% Frank Mittelbach's and David Carlisle's array.sty patches and improves
% the standard LaTeX2e array and tabular environments to provide better
% appearance and additional user controls. As the default LaTeX2e table
% generation code is lacking to the point of almost being broken with
% respect to the quality of the end results, all users are strongly
% advised to use an enhanced (at the very least that provided by array.sty)
% set of table tools. array.sty is already installed on most systems. The
% latest version and documentation can be obtained at:
% http://www.ctan.org/tex-archive/macros/latex/required/tools/


%\usepackage{mdwmath}
%\usepackage{mdwtab}
% Also highly recommended is Mark Wooding's extremely powerful MDW tools,
% especially mdwmath.sty and mdwtab.sty which are used to format equations
% and tables, respectively. The MDWtools set is already installed on most
% LaTeX systems. The lastest version and documentation is available at:
% http://www.ctan.org/tex-archive/macros/latex/contrib/mdwtools/


% IEEEtran contains the IEEEeqnarray family of commands that can be used to
% generate multiline equations as well as matrices, tables, etc., of high
% quality.


%\usepackage{eqparbox}
% Also of notable interest is Scott Pakin's eqparbox package for creating
% (automatically sized) equal width boxes - aka "natural width parboxes".
% Available at:
% http://www.ctan.org/tex-archive/macros/latex/contrib/eqparbox/





% *** SUBFIGURE PACKAGES ***
%\usepackage[tight,footnotesize]{subfigure}
% subfigure.sty was written by Steven Douglas Cochran. This package makes it
% easy to put subfigures in your figures. e.g., "Figure 1a and 1b". For IEEE
% work, it is a good idea to load it with the tight package option to reduce
% the amount of white space around the subfigures. subfigure.sty is already
% installed on most LaTeX systems. The latest version and documentation can
% be obtained at:
% http://www.ctan.org/tex-archive/obsolete/macros/latex/contrib/subfigure/
% subfigure.sty has been superceeded by subfig.sty.



%\usepackage[caption=false]{caption}
%\usepackage[font=footnotesize]{subfig}
% subfig.sty, also written by Steven Douglas Cochran, is the modern
% replacement for subfigure.sty. However, subfig.sty requires and
% automatically loads Axel Sommerfeldt's caption.sty which will override
% IEEEtran.cls handling of captions and this will result in nonIEEE style
% figure/table captions. To prevent this problem, be sure and preload
% caption.sty with its "caption=false" package option. This is will preserve
% IEEEtran.cls handing of captions. Version 1.3 (2005/06/28) and later 
% (recommended due to many improvements over 1.2) of subfig.sty supports
% the caption=false option directly:
%\usepackage[caption=false,font=footnotesize]{subfig}
%
% The latest version and documentation can be obtained at:
% http://www.ctan.org/tex-archive/macros/latex/contrib/subfig/
% The latest version and documentation of caption.sty can be obtained at:
% http://www.ctan.org/tex-archive/macros/latex/contrib/caption/




% *** FLOAT PACKAGES ***
%
%\usepackage{fixltx2e}
% fixltx2e, the successor to the earlier fix2col.sty, was written by
% Frank Mittelbach and David Carlisle. This package corrects a few problems
% in the LaTeX2e kernel, the most notable of which is that in current
% LaTeX2e releases, the ordering of single and double column floats is not
% guaranteed to be preserved. Thus, an unpatched LaTeX2e can allow a
% single column figure to be placed prior to an earlier double column
% figure. The latest version and documentation can be found at:
% http://www.ctan.org/tex-archive/macros/latex/base/



%\usepackage{stfloats}
% stfloats.sty was written by Sigitas Tolusis. This package gives LaTeX2e
% the ability to do double column floats at the bottom of the page as well
% as the top. (e.g., "\begin{figure*}[!b]" is not normally possible in
% LaTeX2e). It also provides a command:
%\fnbelowfloat
% to enable the placement of footnotes below bottom floats (the standard
% LaTeX2e kernel puts them above bottom floats). This is an invasive package
% which rewrites many portions of the LaTeX2e float routines. It may not work
% with other packages that modify the LaTeX2e float routines. The latest
% version and documentation can be obtained at:
% http://www.ctan.org/tex-archive/macros/latex/contrib/sttools/
% Documentation is contained in the stfloats.sty comments as well as in the
% presfull.pdf file. Do not use the stfloats baselinefloat ability as IEEE
% does not allow \baselineskip to stretch. Authors submitting work to the
% IEEE should note that IEEE rarely uses double column equations and
% that authors should try to avoid such use. Do not be tempted to use the
% cuted.sty or midfloat.sty packages (also by Sigitas Tolusis) as IEEE does
% not format its papers in such ways.





% *** PDF, URL AND HYPERLINK PACKAGES ***
%
%\usepackage{url}
% url.sty was written by Donald Arseneau. It provides better support for
% handling and breaking URLs. url.sty is already installed on most LaTeX
% systems. The latest version can be obtained at:
% http://www.ctan.org/tex-archive/macros/latex/contrib/misc/
% Read the url.sty source comments for usage information. Basically,
% \url{my_url_here}.





% *** Do not adjust lengths that control margins, column widths, etc. ***
% *** Do not use packages that alter fonts (such as pslatex).         ***
% There should be no need to do such things with IEEEtran.cls V1.6 and later.
% (Unless specifically asked to do so by the journal or conference you plan
% to submit to, of course. )


% correct bad hyphenation here
\hyphenation{op-tical net-works semi-conduc-tor}


\begin{document}
%
% paper title
% can use linebreaks \\ within to get better formatting as desired
\title{PBJ: A Gnutella Inspired File Sharing System }


% author names and affiliations
% use a multiple column layout for up to three different
% affiliations
\author{\IEEEauthorblockN{Camden Clements}
\IEEEauthorblockA{School of Computing\\
Clemson University\\
Clemson, SC 29632\\
Email: camdenc@gmail.com}
\and
\IEEEauthorblockN{Adam Hodges}
\IEEEauthorblockA{School of Computing\\
Clemson University\\
Clemson, SC 29632\\
Email: hodges8@clemson.edu}
\and
\IEEEauthorblockN{Zach Welch}
\IEEEauthorblockA{School of Computing\\
Clemson University\\
Clemson, SC 29632\\
Email: zwelch@clemson.edu}}

% conference papers do not typically use \thanks and this command
% is locked out in conference mode. If really needed, such as for
% the acknowledgment of grants, issue a \IEEEoverridecommandlockouts
% after \documentclass

% for over three affiliations, or if they all won't fit within the width
% of the page, use this alternative format:
% 
%\author{\IEEEauthorblockN{Michael Shell\IEEEauthorrefmark{1},
%Homer Simpson\IEEEauthorrefmark{2},
%James Kirk\IEEEauthorrefmark{3}, 
%Montgomery Scott\IEEEauthorrefmark{3} and
%Eldon Tyrell\IEEEauthorrefmark{4}}
%\IEEEauthorblockA{\IEEEauthorrefmark{1}School of Electrical and Computer Engineering\\
%Georgia Institute of Technology,
%Atlanta, Georgia 30332--0250\\ Email: see http://www.michaelshell.org/contact.html}
%\IEEEauthorblockA{\IEEEauthorrefmark{2}Twentieth Century Fox, Springfield, USA\\
%Email: homer@thesimpsons.com}
%\IEEEauthorblockA{\IEEEauthorrefmark{3}Starfleet Academy, San Francisco, California 96678-2391\\
%Telephone: (800) 555--1212, Fax: (888) 555--1212}
%\IEEEauthorblockA{\IEEEauthorrefmark{4}Tyrell Inc., 123 Replicant Street, Los Angeles, California 90210--4321}}




% use for special paper notices
%\IEEEspecialpapernotice{(Invited Paper)}




% make the title area
\maketitle


\begin{abstract}
%\boldmath
PBJ is a distributed file sharing system designed on many of the same principles of the gnutella file sharing application.  Rather than route search requests through some central entity like Napster and to a lesser extent BitTorrent, PBJ creates a network of connected users.  Search requests are broadcast between users and files are downloaded directly between them.  This paper details the algorithms behind PBJ, the specifics of its implementation, and a comparison of PBJ to other popular file sharing systems.  Future improvements to PBJ are also discussed.   
\end{abstract}
% IEEEtran.cls defaults to using nonbold math in the Abstract.
% This preserves the distinction between vectors and scalars. However,
% if the conference you are submitting to favors bold math in the abstract,
% then you can use LaTeX's standard command \boldmath at the very start
% of the abstract to achieve this. Many IEEE journals/conferences frown on
% math in the abstract anyway.

% no keywords




% For peer review papers, you can put extra information on the cover
% page as needed:
% \ifCLASSOPTIONpeerreview
% \begin{center} \bfseries EDICS Category: 3-BBND \end{center}
% \fi
%
% For peerreview papers, this IEEEtran command inserts a page break and
% creates the second title. It will be ignored for other modes.
\IEEEpeerreviewmaketitle



\section{Introduction}
Peer to peer (P2P) file sharing is probably one of the most well known and ubiquitous examples of a distributed system.  These systems allow users to search for and download files stored on the machines of other users.  Systems such as Napster, BitTorrent, and Gnutella are used by millions of people every day to download files.  While the content on these systems is not always legal (eg. file sharing of media such as music and movies), P2P file sharing networks are..... File sharing is an ideal case study of distributed systems, and highlight many of the design decisions faced when developing such systems. P2P systems can feature a number of different network topographies that range from having a central server storing all the files (Napster) to a distributed system with almost no centralized elements (Gnutella).  In the sections that follow we document and describe PBJ, our P2P file sharing system influenced by Gnutella.

\subsection{Background}
\subsubsection{Napster}
A defining characteristic of a P2P file sharing application is the kind of network topology it employs.  Napster, one of the first file sharing networks, was a highly centralized system.  All  online Napster users connected to a central server.  Each user would have a folder designated for sharing audio files (only mp3s were transferred on Napster).  This central server would handle a user request by searching for related files in the share folders of other users.  The central server would then report a list of relevant files and their locations that the user could choose to download.  Any files the requesting user chose to download would directly connect with the associated machine for actual file transfer.  The ``flaw'' of Napster's system is that the central server is appropriately named; if the server goes down, then the network ceases to exist because no requests can be processed.  This is exactly what happened in 2001 when the inevitable flood of copyright infringement lawsuits were filed against Napster.  File sharing applications developed after Napster�s shut down attempted to avoid this problem by making the system less centralized. 
\subsubsection{BitTorrent}
 Most file sharing today uses the BitTorrent protocol.  BitTorrent differs from other file sharing systems in that where as Napster and gnutella transfer entire files between nodes, BitTorrent downloads parts of the file from multiple sources.  BitTorrent works by having a user obtain a torrent file (usually by downloading it from a website).  The local BitTorrent client interprets the contents of the torrent file and connects with a tracker, a server with information about how to find the file. The tracker finds seeders, other BitTorrent clients with a local copy of the file to be transferred. The file also identifies the swarm, a set of clients with a portion of the file, usually in the process of downloading it themselves.  The searching client downloads from these sources simultaneously to build the desired file.  BitTorrent is especially useful for downloading popular files, since there will be a large pool of seeders and a large swarm.  While BitTorrent is clearly more distributed than Napster, since their are multiple central servers instead of a single large server for the entire application.  However, if the tracker(s) in a torrent file are taken down, there is no way that torrent file can be used to obtain the actual file.     
 \subsubsection{Gnutella}
 Gnutella (a portmanteau of GNU, the free software project, and Nutella, the chocolate and hazelnut spread) is a decentralized network of interconnected users running a gnutella client called a node.  Search requests are broadcast to all of the searching node�s neighbor clients (which gnutella terms �peers�), which in turn broadcast the request to all of their peers as well.  This goes on until the file is found or the request goes a specified number of peer hops without finding a file (much like the time to live field in an IP packet). This does allow for the possibility that a search request may fail to return a file in the network. Later versions of gnutella�s network introduced the concept of ultra nodes and ultra peers. Each ultra node is connected to a network of regular nodes and a large number of other ultra nodes, each with its own network of nodes, making gnutella a network of networks.  The reasons for this change is that it many more nodes can be reached from any one node in a few hops, making searching more efficient and the system more scalable.  Nodes keep track of nodes and ultra nodes they have previously connected with and attempt to reconnect to the network through these nodes.  When connecting for the first time, these nodes must attempt to a set of nodes guaranteed to be in the gnutella network.  This means that despite the highly decentralized nature of gnutella, a few centralized elements remain.

% An example of a floating figure using the graphicx package.
% Note that \label must occur AFTER (or within) \caption.
% For figures, \caption should occur after the \includegraphics.
% Note that IEEEtran v1.7 and later has special internal code that
% is designed to preserve the operation of \label within \caption
% even when the captionsoff option is in effect. However, because
% of issues like this, it may be the safest practice to put all your
% \label just after \caption rather than within \caption{}.
%
% Reminder: the "draftcls" or "draftclsnofoot", not "draft", class
% option should be used if it is desired that the figures are to be
% displayed while in draft mode.
%
%\begin{figure}[!t]
%\centering
%\includegraphics[width=2.5in]{myfigure}
% where an .eps filename suffix will be assumed under latex, 
% and a .pdf suffix will be assumed for pdflatex; or what has been declared
% via \DeclareGraphicsExtensions.
%\caption{Simulation Results}
%\label{fig_sim}
%\end{figure}

% Note that IEEE typically puts floats only at the top, even when this
% results in a large percentage of a column being occupied by floats.


% An example of a double column floating figure using two subfigures.
% (The subfig.sty package must be loaded for this to work.)
% The subfigure \label commands are set within each subfloat command, the
% \label for the overall figure must come after \caption.
% \hfil must be used as a separator to get equal spacing.
% The subfigure.sty package works much the same way, except \subfigure is
% used instead of \subfloat.
%
%\begin{figure*}[!t]
%\centerline{\subfloat[Case I]\includegraphics[width=2.5in]{subfigcase1}%
%\label{fig_first_case}}
%\hfil
%\subfloat[Case II]{\includegraphics[width=2.5in]{subfigcase2}%
%\label{fig_second_case}}}
%\caption{Simulation results}
%\label{fig_sim}
%\end{figure*}
%
% Note that often IEEE papers with subfigures do not employ subfigure
% captions (using the optional argument to \subfloat), but instead will
% reference/describe all of them (a), (b), etc., within the main caption.


% An example of a floating table. Note that, for IEEE style tables, the 
% \caption command should come BEFORE the table. Table text will default to
% \footnotesize as IEEE normally uses this smaller font for tables.
% The \label must come after \caption as always.
%
%\begin{table}[!t]
%% increase table row spacing, adjust to taste
%\renewcommand{\arraystretch}{1.3}
% if using array.sty, it might be a good idea to tweak the value of
% \extrarowheight as needed to properly center the text within the cells
%\caption{An Example of a Table}
%\label{table_example}
%\centering
%% Some packages, such as MDW tools, offer better commands for making tables
%% than the plain LaTeX2e tabular which is used here.
%\begin{tabular}{|c||c|}
%\hline
%One & Two\\
%\hline
%Three & Four\\
%\hline
%\end{tabular}
%\end{table}


% Note that IEEE does not put floats in the very first column - or typically
% anywhere on the first page for that matter. Also, in-text middle ("here")
% positioning is not used. Most IEEE journals/conferences use top floats
% exclusively. Note that, LaTeX2e, unlike IEEE journals/conferences, places
% footnotes above bottom floats. This can be corrected via the \fnbelowfloat
% command of the stfloats package.

\section{Methodology}

PBJ takes the key qualities of gnutella and uses them as a starting point, rather than implement a version of the gnutella client.  LIke gnutella, PBJ contains a network of ultra nodes.  In PBJ, each ultra node has a set of nodes.  Note that ultra nodes are also nodes in PBJ.  If there are k nodes per ultra node, the first node to be added to the network will become an ultra node, the next k nodes will be added to the ultra node.  The k+1th node will then become a new ultra node.  Ultra nodes in PBJ are given a unique ID.  This unique ID is a vital element to building the network. The basic algorithm for creation of the network builds an outward spiral of ultra peers.  While this alone would be wildly inefficient in terms of searching and stability, new ultra nodes also connect to ultra nodes deeper within the spiral.  This allows search requests to quickly move between otherwise distant ultra nodes. More specifically, the first node created will be given an ID of 0, the second an ID of 1, etc.  When a node is created with ID Y, it will attempt to connect to all nodes X, 0 < X < Y, where Y - X = 2Z, Z being an even integer.  So for example, node 16 would connect with node 15 (16 - 15 = 1 = 20), node 12 (16 - 12 = 4 = 22), and node 0 (16 - 0 = 16 = 24). To make talking about the network easier, we delineate between kinds of hops between nodes.  For example, a hop from 5 to 6 is a 1-hop, where a hop from 5 to 9 is a 4-hop etc.  

The PBJ network structure presents several clear benefits.  The first clear gain is that the PBJ network is that there is a high level of redundancy in the connections.  This means that the nodes are connected in such a way that it is easy for search requests to reach most if not all of the nodes on the network in a few hops.  Currently, building the network is done through a predefined gateway at a predefined IP address and port.  The gateway keeps track of all the ultrapeers in the system.  When a new node tries to join the system, the gateway makes several decisions based on the current status of the PBJ network.  First, the gateway decides if the new node will become an ultra node.  If so, it will give the new ultra node a list of other ultra nodes to connect to.  If the new node is not an ultra peer, the gateway will provide information on the appropriate ultra peer sub network to join. These decisions by the gateway essentially dictate how the network is built. We devised two separate algorithms for building the network; the one we actually implemented is described here and the other is detailed in the section �Future Work�.  As implemented, the first node to enter the network becomes ultra peer 0.  If, for example, there are three nodes per ultra node, the next three nodes to connect would become node 0�s sub network.  The fifth node to join the network would become ultra node 1 and would connect to ultra node 0. In this scheme of building the network, each ultra node�s sub network is filled before the next ultra node is created.  

To share files, a user must specify a share directory.  All files and sub directories in their share directory will be available to the network for download.  When a user wants to search, they enter a keyword and submit a search request.  The search request will be sent to the searching node�s ultra node.  The ultra node will send the search request to all the other peers in its sub network before broadcasting it to its connected ultra peers.  Search requests consist of four major parts - the searching node�s address, the keyword, a search id, and the time to live (ttl).  The ttl is initially set to a positive integer.  Each time a search request is passed between ultra peers, the request ttl is decremented.  Notice that ttl only limits the number of hops among ultra peers, passing a request within an ultra node and its sub network has no effect.  If an ultra node receives a search request with a ttl of 0, it will send the request to its sub network and delete the request.  If a keyword match is found, the node with the desired file sends a message to the node specified in the search request with its address and the path of the desired file relative to the share folder.  As these acknowledge messages come in to the search node, it displays to the user a list of all the files found on the network.  The user can then choose one or more files to download.  The searching node directly connects to the node with the selected file and downloads it.  The search id is sent as an attempt to significantly lessen the search request traffic on the network.  The structure of the PBJ network means there are many different paths between two ultra nodes.  While this is very useful in keeping the network stable and allowing search requests to quickly reach a large number of files, it also means that the same ultra node will receive and process a request multiple times, sending the message out to all of its ultra peers, who have already received the request.  This type of network behavior is extremely inefficient.  PBJ attempts to fix this problem in part by having a unique pair of values for each search request.  One of these values is the ultra node id the search request originated from and the other is local to each ultra node and represents the number of previous requests sent out by the ultra node.  The first request sent from ultra node 6 would have an id pair of (6,1), the second would have (6,2) etc.  Each ultra node keeps track of the last several request ids received and if the request id of a new search request matches one in the list, that request is ignored.  This scheme vastly cuts down the amount of redundant request traffic and makes the system more efficient as a whole.     

An area of contention during development of PBJ was how to correctly handle ultra peer disconnection.  Various solutions were proposed, including updating the gateway, electing a node from the sub network to be a new ultra node, and the Ostrich approach (ignoring it).  We eventually decided to implement a system of letting the gateway know when an ultra node goes down.  Every time the gateway is about to add a new node to the network, it pings its ultra peers.  If one or more does not respond, it marks those as missing, and inserts ultra nodes into these positions until they are all filled, and then fills them The point here is to ensure important connections in the network are rebuilt before the ultra nodes are filled.  In addition, ultra nodes will occasionally ping their ultra peers and sub network.  If an ultra node does not get a response from an ultra peer, it removes it from the list of connected ultra peers.  If an ultra peer cannot connect to a peer in its sub network, it removes that peer from its list and notifies the gateway that a spot has opened up in its sub network.  If a node in a sub network cannot reach its ultra peer, it re enters the network through the gateway.  This ensures that all lost nodes will be replaced as new ones are added to the system, and that the network will not become fractured with high node turnover.

\section{Results}

There are a variety of improvements that PBJ could benefit from in future distributions.  One previously discussed is changing how nodes join the system.  The current gateway method functions well, but its centralized nature is hardly ideal for a system whose design so highlights decentralization.  Several alternatives have been discussed. The most likely implementation would be to place the burden of network building on the existing ultra nodes.  A node would keep track of its previously connected ultra nodes and attempt to connect to these peers.  Depending on its current status, the ultra peer might add the new users or decide to push it towards higher valued ultra peers to attempt placement.  Eventually it would either get placed in a sub network or reach the ultra peer with the highest id and become the new highest ultra peer.  If none of the previously connected nodes are currently connected to the network, a range of options could be implemented to ensure the user gets into the network including but not limited to : keeping the gateway as a backup, having ultra nodes at known URLs, placing the burden on the user to find the information. 



Several different specific algorithms for how exactly the gateway builds the grid were discussed and implemented.  The basic algorithm, described above, creates a new ultra node and then fills that ultra node�s sub network before creating a new ultra peer.  While this algorithm works acceptably, it ignores several observations about PBJ�s network.  In PBJ, the number of hops it takes to get from node A to node B is highly dependent on the current number of ultra peers in the system.  As a simple example, the optimal number of hops from ultra peers 0 to 63 is 5 hops (0-16-32-48-47-63) if ultra node 64 does not exist and 2 hops (0 - 64-63) if ultra node 64 does exist.  Indeed, a node whose largest hop was one of the last nodes to be added has the optimal network structure for searching.  Our algorithm for taking advantage of this trait slightly alters how the network is built by the gateway.  The basic idea behind this algorithm is to have an ultra peer back bone in place first with empty sub networks and then fill in the sub networks.  When creating a new network, the first eight (eight is half of sixteen) nodes into the system will become the first 8 ultra peers.  The next nodes to join the system will join these ultra peers� sub networks.  When all eight sub networks have been filled, the gateway then add the next nodes as ultra peers in the system until there are thirty two ultra peers (thirty two is half of sixty four) in the system.  This continues so that when the network is full, it grows to half the size of the next even power of 2 before filling in the ultra peers.  This ensures that the majority of the peers are close together but have the ability to move farther because there is an ultra peer structure in place to facilitate such movement. Another potential improvement would be to choose ultra peers based on network quality.  Ultra peers have to handle the vast majority of the network traffic.  It then makes sense that the machines with the strongest network connection should be ultra peers.  This would help stabilize the network and improve quality.

Simple things such as an improved GUI and security/privacy features could also be added to make the system more user friendly.  Another concern we have is attempting to mitigate is leeching, the practice where by users of a file sharing network do not share any files to be downloaded but still use the network to get files.

\section{Analysis}



\section{Conclusion}
The conclusion goes here.




% conference papers do not normally have an appendix


% use section* for acknowledgement
\section*{Acknowledgment}


The authors would like to thank...





% trigger a \newpage just before the given reference
% number - used to balance the columns on the last page
% adjust value as needed - may need to be readjusted if
% the document is modified later
%\IEEEtriggeratref{8}
% The "triggered" command can be changed if desired:
%\IEEEtriggercmd{\enlargethispage{-5in}}

% references section

% can use a bibliography generated by BibTeX as a .bbl file
% BibTeX documentation can be easily obtained at:
% http://www.ctan.org/tex-archive/biblio/bibtex/contrib/doc/
% The IEEEtran BibTeX style support page is at:
% http://www.michaelshell.org/tex/ieeetran/bibtex/
%\bibliographystyle{IEEEtran}
% argument is your BibTeX string definitions and bibliography database(s)
%\bibliography{IEEEabrv,../bib/paper}
%
% <OR> manually copy in the resultant .bbl file
% set second argument of \begin to the number of references
% (used to reserve space for the reference number labels box)
\begin{thebibliography}{1}

\bibitem{IEEEhowto:kopka}
H.~Kopka and P.~W. Daly, \emph{A Guide to \LaTeX}, 3rd~ed.\hskip 1em plus
  0.5em minus 0.4em\relax Harlow, England: Addison-Wesley, 1999.

\end{thebibliography}




% that's all folks
\end{document}


